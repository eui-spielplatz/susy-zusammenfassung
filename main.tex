\documentclass[10pt,a4paper]{article}
\usepackage[latin1]{inputenc}
\usepackage{amsmath}
\usepackage{amsfonts}
\usepackage{amssymb}
\usepackage{graphicx}
\usepackage{hyperref}
\usepackage{color}
\usepackage[left=2.50cm, right=2.00cm, top=2.50cm, bottom=2.50cm]{geometry}
\usepackage[txtcentered=true, height=40pt, width=70pt]{thumbs}
\usepackage[german]{babel}
\author{Florian Euchner, Stefan K�bel, Chong Shen, Jan Frederik Dick}

\newcommand{\fancythumb}[2]{
	\addthumb{#1}{\large\sffamily\textbf{\space\space#1\vspace{5pt}}}{white}{#2}
}

\newcommand{\fancyformula}[2]{
	\small
	\raggedright\sffamily\textbf{#1}
	#2
}

\pagenumbering{arabic}

\begin{document}
	\section{Signale und Systeme}
	\fancythumb{S\&S}{teal}
	\subsection{Signal}
	Ein Signal ist eine Funktion x(t). Sie liefert f�r jedes Funktionsargument t ein ein Funktionswert x(t).
	\subsection{Klassifikation von Signalen}
	\begin{itemize}
	\item gerade/ungerade
	\item real/imagin�r/komplex
	\item periodisch/nicht periodisch
	\item einkanalig
	\item mehrkanalig
	\item eindimensional
	\item mehrdimensional
	\item zeitkontinuierlich
	\item zeitdiskret
	\item wertkontinuierlich
	\item wertdiskret
	\item analog
	\item digital
	\end{itemize}
	\subsection{Zeitkontinuierliche Signale}
	\subsection{Zeitdiskrete Signale}
	\subsection{System}
	\subsection{Klassifikation von Systemen}
	\subsection{Parallel- und Kettenschaltung von Systemen}
	\newpage
	\section{Zeitkontinuierliche LTI-Systeme im Zeitbereich}
	\fancythumb{ZK-t}{red}
	\newpage
	\section{Zeitdiskrete LTI-Systeme im Zeitbereich}
	\fancythumb{ZD-t}{magenta}
	\newpage
	\section{Zeitkontinuierliche LTI-Systeme im Frequenzbereich}
	\fancythumb{ZK-f}{blue}
	\newpage
	\section{Zeitdiskrete LTI-Systeme im Frequenzbereich}
	\fancythumb{ZD-f}{violet}
	\newpage
	\section{Analyse von Signalen und LTI-Systemen in der komplexen Ebene}
	\fancythumb{$\mathbb{C}$}{olive}
	\newpage
	\section{Allgemeinere Signale und Systeme}
	\fancythumb{Allg. S\&S}{black}
\end{document}