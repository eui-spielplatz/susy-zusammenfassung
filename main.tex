\documentclass[10pt,a4paper]{article}
\usepackage[utf8]{inputenc}
\usepackage{titlesec}
\usepackage{amsfonts}
\usepackage{graphicx}
\usepackage{hyperref}
\usepackage{amsmath}
\usepackage{amssymb}
\usepackage{color}
\usepackage[left=2.50cm, right=2.00cm, top=2.50cm, bottom=2.50cm]{geometry}
\usepackage[txtcentered=true, height=40pt, width=70pt]{thumbs}
\usepackage[german]{babel}
\author{Florian Euchner, Stefan Köbel, Chong Shen, Jan Frederik Dick}

\newcommand{\fancythumb}[2]{
	\addthumb{#1}{\large\sffamily\textbf{\space\space#1\vspace{5pt}}}{white}{#2}
}

\pagenumbering{arabic}
\titleformat*{\section}{\sffamily\huge\bfseries}
\titleformat*{\subsection}{\sffamily\Large\bfseries}
\titleformat*{\subsubsection}{\sffamily\large\bfseries}

\begin{document}
\section*{Signale und Systeme}
\fancythumb{S\&S}{teal}
\subsection*{Signal}
Ein Signal ist eine \textbf{Funktion} $x(t)$. Sie liefert für jedes Funktionsargument $t$ einen Funktionswert $x(t)$.
\subsection*{Klassifikation von Signalen}
\begin{itemize}
	\item \textbf{gerade/ungerade}: $x(t)=x_g(t)+x_u(t)$, wobei $x_g(-t)=x_g(t)$ und $x_u(-t)=-x_u(t)$, $x_u(0)=0$
	\item \textbf{reell/imaginär/komplex}: $x(t)=x_r(t)+jx_i(t)$ und $x^*(t)=x_r(t)-jx_i(t)$, wobei $x^*(t)$ komplex konjugiert zu $x(t)$, Realteil: $x_r(t)=\dfrac{1}{2}(x(t)+x^*(t))$ und Imaginärteil: $x_i(t)=\dfrac{1}{2j}(x(t)-x^*(t))$ 
	\item \textbf{periodisch}: es existiert $T$, sodass $x(t+T)=x(t)$ für alle $t$; Kleinstes $T$ heißt dann Periode
	\item \textbf{einkanalig}: ein Funktionswert, z.B. $x(t)$
	\item \textbf{mehrkanalig}: mehrere Funktionswerte, z.B.
	$\begin{bmatrix}
		x_1(t) \\ 
		x_2(t)
	\end{bmatrix}$
	\item \textbf{eindimensional}: ein Funktionsargument, z.B. $x(t)$
	\item \textbf{mehrdimensional}:mehrere Funktionsargumente, z.B. $x(t,x,y)$
	\item \textbf{zeitkontinuierlich}: zeitkontinuierliches Funktionsargument, $x(t)$ mit $t \in \mathbb R$
	\item \textbf{zeitdiskret}: zeitdiskretes Funktionsargument, $x(n)$ mit $n \in \mathbb Z$
	\item \textbf{wertkontinuierlich}: kontinuierlicher Funktionswert
	\item \textbf{wertdiskret}: diskreter Funktionswert
	\item \textbf{analog}: zeitkontinuierliche und wertkontinuierlich
	\item \textbf{digital}: zeitdiskret und wertdiskret
\end{itemize}
\subsection*{Zeitkontinuierliche Signale}
\begin{itemize}
	\item Sprungfunktion
	\item Vorzeichenfunktion
	\item Rechteckfunktion
	\item Reelles oder komplexes Sinussignal
	\item Exponentialsignal
	\item verschobenes Rechteck
	\item Dreieck
\end{itemize}
\subsection*{Zeitdiskrete Signale}
\begin{itemize}
	\item Sprungfunktion
	\item Vorzeichenfunktion
	\item Rechteckfunktion
	\item Reelles oder komplexes Sinussignal
	\item Exponentialsignal
\end{itemize}
\subsection*{System}
	Das System hat einen \textbf{Operator}, der eine Funktion auf eine andere Funktion abbildet. Ein System heißt \textbf{zeitkontinuierlich, zeitdiskret, analog, digital}, wenn alle beteiligten Signale so heißen.
\subsection*{Klassifikation von Systemen}
\begin{itemize}
	\item Gedächtnis
	\item Kausalität
	\item Zeitvarianz
	\item Linearität
	\item BIBO-Stabilität
\end{itemize}
\subsection*{Parallel- und Kettenschaltung von Systemen}
\begin{itemize}
	\item \textbf{Parallelschaltung} $y(t)=T_1(x(t))+T_2(x(t))+...+T_N(x(t))$
	\item \textbf{Kettenschaltung} 	$y(t)=T_N(...T_2(T_1(x(t)))...)$
\end{itemize}

\newpage
\section*{Zeitkontinuierliche LTI-Systeme im Zeitbereich}
\fancythumb{ZK-t}{red}

\newpage
\section*{Zeitdiskrete LTI-Systeme im Zeitbereich}
\fancythumb{ZD-t}{magenta}

\newpage
\section*{Zeitkontinuierliche LTI-Systeme im Frequenzbereich}
\fancythumb{ZK-f}{blue}

\newpage
\section*{Zeitdiskrete LTI-Systeme im Frequenzbereich}
\fancythumb{ZD-f}{violet}

\newpage
\section*{Analyse von Signalen und LTI-Systemen in der komplexen Ebene}
\fancythumb{$\mathbb{C}$}{olive}

\newpage
\section*{Allgemeinere Signale und Systeme}
\fancythumb{Allg. S\&S}{black}
\end{document}